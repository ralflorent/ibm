% Individual-Based Modeling (IBM)
%
% Advanced Project I - Jacobs University Bremen
% Supervisor: Dr. Stefan Kettemann
%
% Created on June 19, 2019
%
% Authors:
%   Ralph Florent <r.florent@jacobs-university.de>
%   Davi Tavares <davi.tavares@leibniz-zmt.de>
%   Agostino Merico <a.merico@jacobs-university.de>
%
% Theory review for the documentation

% ==============================================================================
% START: Theoretical background
% ==============================================================================

\section{Theoretical Background}

\subsection{Waterbirds and Environmental Factors}
% Tropical coastal lagoons are shallow aquatic ecosystems located at the boundary between terrestrial and marine environments. The high environmental heterogeneity of coastal lagoons, in both temporal and spatial scales, provides habitats for aquatic bird species with different ecological needs. Previous studies have documented the sensitivity of waterbirds to habitat alterations and have highlighted the importance of management strategies, especially under the current context of rapid environmental change and biodiversity loss. According to [X], the habitat factors to account for when implementing strategy management are: water depth, vegetation height, lagoon size, water salinity, water pH, grazing pressure, and distance from human settlements.

\subsection{Agent-Based Modeling}
Agent-Based Models are computational simulation models that involve many discrete agents \cite{ago2019abm}. This computational simulation is usually based on intense processings and algorithmic calculations due to the fact the typical context in which the ABM is used is to study the collective behaviour of large number of components or agents.

An \emph{agent} is a component or an entitiy of the system and contains usually the following properties: internal states, spatial locations, interaction with the environment, interaction with each other, behaviour rules, adaptation and evolution. Depending on the goal of the ABM, some additional properties may or not be incorporated into the model. For instance, certain agents can be attributed the role of \emph{central controllers}.

The code implementation of an ABM can particularly be as heavy as its model complexity increases. Hence, it is viable to start off with some uncomplicated settings and assumptions in order to favour a straightforward analysis of the results that are obtained after running the simulation. Afterwards, one can subsequently tranform the model by adding more complexities. On the other hand, from a programming point of view, the code maintenance and organization are a relevant factor that contributes to debug relatively faster as the amount of coding increases.

% ==============================================================================
% END: Theoretical background
% ==============================================================================