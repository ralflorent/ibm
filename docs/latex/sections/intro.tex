% Individual-Based Modeling (IBM)
%
% Advanced Project I - Jacobs University Bremen
% Supervisor: Dr. Stefan Kettemann
%
% Created on May 29, 2019
%
% Authors:
%   Ralph Florent <r.florent@jacobs-university.de>
%   Davi Tavares <davi.tavares@leibniz-zmt.de>
%   Agostino Merico <a.merico@jacobs-university.de>
%
% Introduction for the documentation

% ==============================================================================
% START: Introduction
% ==============================================================================

\section{Introduction}
Complex Systems is a field of science studying how individual components of a system give rise to collective behaviours and how the system interacts with its environment \cite{ago2019abm}. One of the approaches to study a system interaction with its environment is through Agent-Based Modeling (ABM), a generalized framework for modeling and simulating dynamical systems.

In our case, we intend to study the habitat use by waterbirds in coastal lagoons of the tropics. ABM, besides being a computational simulation, is the closest modeling assumptions capable of providing a deeper understanding and interpretability of such a system.

This project report outlines the different steps to achieve a Virtual Environment (VE) using an ABM technique to characterize the waterbirds behaviours, adaptation and evolution within a set of habitats with distinguishable properties. These steps are a reference to the Python code implementation of this VE, which serves as a demonstration basis to run and simulate the ABM. Finally, the results are analyzed and discussed in accordance with the algorithmic methods, the content structure, and the workflow scheme that are derived mostly from the representative traits of each component of the system.

% ==============================================================================
% END: Introduction
% ==============================================================================